\documentclass[paper=a4, fontsize=11pt]{scrartcl} % A4 paper and 11pt font size
\usepackage[T1]{fontenc} % Use 8-bit encoding that has 256 glyphs
\usepackage[english]{babel} % English language/hyphenation
\usepackage{multirow}
\usepackage{graphicx}
\usepackage{amsmath,amsfonts,amsthm} % Math packages
\usepackage{sectsty} % Allows customizing section commands
\allsectionsfont{\centering \normalfont\scshape} % Make all sections centered, the default font and small caps

\usepackage{fancyhdr} % Custom headers and footers
\pagestyle{fancyplain} % Makes all pages in the document conform to the custom headers and footers
\fancyhead{} % No page header - if you want one, create it in the same way as the footers below
\fancyfoot[L]{} % Empty left footer
\fancyfoot[C]{} % Empty center footer
\fancyfoot[R]{\thepage} % Page numbering for right footer
\renewcommand{\headrulewidth}{0pt} % Remove header underlines
\renewcommand{\footrulewidth}{0pt} % Remove footer underlines
\setlength{\headheight}{13.6pt} % Customize the height of the header

%\numberwithin{equation}{section} % Number equations within sections (i.e. 1.1, 1.2, 2.1, 2.2 instead of 1, 2, 3, 4)
%\numberwithin{figure}{section} % Number figures within sections (i.e. 1.1, 1.2, 2.1, 2.2 instead of 1, 2, 3, 4)
%\numberwithin{table}{section} % Number tables within sections (i.e. 1.1, 1.2, 2.1, 2.2 instead of 1, 2, 3, 4)

%\setlength\parindent{0pt} % Removes all indentation from paragraphs - comment this line for an assignment with lots of text

\newcommand{\horrule}[1]{\rule{\linewidth}{#1}} % Create horizontal rule command with 1 argument of height

\title{	
\normalfont \normalsize 
\textsc{TMA4220 Numerical Solution of Differential Equations by element methods} \\ [25pt] % Your university, school and/or department name(s)
\horrule{0.5pt} \\[0.4cm] % Thin top horizontal rule
\huge Part 2: Vibration analysis \\ % The assignment title
\horrule{2pt} \\[0.5cm] % Thick bottom horizontal rule
}

\author{Candidate numbers} % Your name

\date{\normalsize\today} % Today's date or a custom date

\begin{document}
\maketitle

\section*{Abstract}

\section*{Introduction}
With the displacement of spatial points in $x_1$- and $x_2$-direction represented as
\begin{equation*}
\boldsymbol{u} = \begin{bmatrix}
u_1 \\ u_2
\end{bmatrix}
\end{equation*}
the strain on each point is
\begin{equation*}
\begin{bmatrix}
\epsilon_{11} \\
\epsilon_{22} \\
\epsilon_{12}
\end{bmatrix}
=
\begin{bmatrix}
\frac{\partial u_1}{\partial x_1} \\
\frac{\partial u_2}{\partial x_2} \\
\frac{\partial u_1}{\partial x_2}+\frac{\partial u_2}{\partial x_1}
\end{bmatrix}
\end{equation*}
as many notes on linear elasticity will let you know.\footnote{Note however that this is the form usually called engineer strain.} 

The connection between the strain $\epsilon$ and the stress $\sigma$ is
\begin{align*}
\begin{bmatrix}
\sigma_{11} \\ \sigma_{22} \\ \sigma_{12}
\end{bmatrix}
&= \frac{E}{1-\nu^2}\begin{bmatrix}
1 & \nu & 0 \\
\nu & 1 & 0 \\
0 & 0 & \frac{1-\nu}{2}
\end{bmatrix}
\begin{bmatrix}
\epsilon_{11} \\ \epsilon_{22} \\ \epsilon_{12}
\end{bmatrix} \\
\boldsymbol{\sigma} &= C\boldsymbol{\epsilon}
\end{align*}
where $E$ is the Young's modulus and $\nu$ is the Poisson's ratio. Young's modulus characterises the solids stiffness, i.e. large $E$ means that you need a large force to deform the solid. Poisson's ratio  is the ratio between the strains in $x_1$- and $x_2$-direction when submitting the solid to a stress in only one of the directions. To clarify: Consider a stress in only the $x_1$-direction direction\footnote{$\sigma_{11}$ being the only stress different from zero.}, then $\nu>0$ will say that the solid contracts in the $x_2$-direction and elongates in the $x_1$-direction. Worth noting is that $\nu<0$ is possible, and some materials actually have this property. Weird.

During the course of this analysis we will let $E$ and $\nu$ completely describe a solid's properties. No stress or strain due to difference in temperature.

\end{document}