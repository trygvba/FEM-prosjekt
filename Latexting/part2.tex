\documentclass[paper=a4, fontsize=11pt]{scrartcl} % A4 paper and 11pt font size
\usepackage[T1]{fontenc} % Use 8-bit encoding that has 256 glyphs
\usepackage[english]{babel} % English language/hyphenation
\usepackage{multirow}
\usepackage{graphicx}
\usepackage{amsmath,amsfonts,amsthm} % Math packages
\usepackage{sectsty} % Allows customizing section commands
\allsectionsfont{\centering \normalfont\scshape} % Make all sections centered, the default font and small caps

\usepackage{fancyhdr} % Custom headers and footers
\pagestyle{fancyplain} % Makes all pages in the document conform to the custom headers and footers
\fancyhead{} % No page header - if you want one, create it in the same way as the footers below
\fancyfoot[L]{} % Empty left footer
\fancyfoot[C]{} % Empty center footer
\fancyfoot[R]{\thepage} % Page numbering for right footer
\renewcommand{\headrulewidth}{0pt} % Remove header underlines
\renewcommand{\footrulewidth}{0pt} % Remove footer underlines
\setlength{\headheight}{13.6pt} % Customize the height of the header

%\numberwithin{equation}{section} % Number equations within sections (i.e. 1.1, 1.2, 2.1, 2.2 instead of 1, 2, 3, 4)
%\numberwithin{figure}{section} % Number figures within sections (i.e. 1.1, 1.2, 2.1, 2.2 instead of 1, 2, 3, 4)
%\numberwithin{table}{section} % Number tables within sections (i.e. 1.1, 1.2, 2.1, 2.2 instead of 1, 2, 3, 4)

%\setlength\parindent{0pt} % Removes all indentation from paragraphs - comment this line for an assignment with lots of text

\newcommand{\horrule}[1]{\rule{\linewidth}{#1}} % Create horizontal rule command with 1 argument of height

\title{	
\normalfont \normalsize 
\textsc{TMA4220 Numerical Solution of Differential Equations by element methods} \\ [25pt] % Your university, school and/or department name(s)
\horrule{0.5pt} \\[0.4cm] % Thin top horizontal rule
\huge Part 2: Vibration analysis \\ % The assignment title
\horrule{2pt} \\[0.5cm] % Thick bottom horizontal rule
}

\author{Candidate numbers} % Your name

\date{\normalsize\today} % Today's date or a custom date

\begin{document}
\maketitle

\section*{Abstract}

\section*{Introduction}
With the displacement of spatial points in $x_1$- and $x_2$-direction represented as
\begin{equation*}
\boldsymbol{u} = \begin{bmatrix}
u_1 \\ u_2
\end{bmatrix}
\end{equation*}
the strain on each point is
\begin{equation*}
\begin{bmatrix}
\varepsilon_{11} \\
\varepsilon_{22} \\
\varepsilon_{12}
\end{bmatrix}
=
\begin{bmatrix}
\frac{\partial u_1}{\partial x_1} \\
\frac{\partial u_2}{\partial x_2} \\
\frac{\partial u_1}{\partial x_2}+\frac{\partial u_2}{\partial x_1}
\end{bmatrix}
\end{equation*}
as many notes on linear elasticity will let you know.\footnote{Note however that this is the form usually called engineer strain.} 

The connection between the strain $\varepsilon$ and the stress $\sigma$ is
\begin{align*}
\begin{bmatrix}
\sigma_{11} \\ \sigma_{22} \\ \sigma_{12}
\end{bmatrix}
&= \frac{E}{1-\nu^2}\begin{bmatrix}
1 & \nu & 0 \\
\nu & 1 & 0 \\
0 & 0 & \frac{1-\nu}{2}
\end{bmatrix}
\begin{bmatrix}
\varepsilon_{11} \\ \varepsilon_{22} \\ \varepsilon_{12}
\end{bmatrix} \\
\boldsymbol{\sigma} &= C\boldsymbol{\varepsilon}
\end{align*}
where $E$ is the Young's modulus and $\nu$ is the Poisson's ratio. Young's modulus characterises the solid's stiffness, i.e. large $E$ means that you need a large force to deform the solid. Poisson's ratio  is the ratio between the strains in $x_1$- and $x_2$-direction when submitting the solid to a stress in only one of the directions. To clarify: Consider a stress in only the $x_1$-direction direction\footnote{$\sigma_{11}$ being the only stress different from zero.}, then $\nu>0$ will say that the solid contracts in the $x_2$-direction and elongates in the $x_1$-direction. Worth noting is that $\nu<0$ is possible, and some materials actually have this property. Weird.
During the course of this analysis we will let $E$ and $\nu$ completely describe a solid's properties. No stress or strain due to difference in temperature.

\subsection*{The equation and a variational formulation}
In continuum mechanics
\begin{equation}
\label{vibDiff}
\rho \ddot{\boldsymbol{u}} = \nabla \sigma(\boldsymbol{u})
\end{equation}
describes the spatial displacement due to internal stresses alone. It is the extension of Hook's law and Newton's second law into a continuous medium. So rather than total mass we look at mass density, here denoted $\rho$.

To eventually arrive at variational formulation of (\ref{vibDiff}) we should spend some time with the right hand side of this equation to better understand it. For that reason we will take a slight detour and start our journey by considering the equation
\begin{align}
\label{linEl}
\nabla \sigma(\boldsymbol{u}) &= - \boldsymbol{f} \\
\left[\frac{\partial}{\partial x_1},  \frac{\partial}{\partial x_2}\right]\begin{bmatrix}
\sigma_{11} & \sigma_{12} \\
\sigma_{12} & \sigma_{22}
\end{bmatrix} &= -[f_1,f_2] \notag
\end{align}
working on some domain $\Omega$,
together with some boundary conditions, which in its general form looks a little something like
\begin{align*}
\boldsymbol{u} &= \boldsymbol{g}, \quad \text{on } \partial \Omega_D \\
\sigma(\boldsymbol{u})\cdot \boldsymbol{\hat{n}} &= \boldsymbol{h}, \quad \text{on } \partial \Omega_N
\end{align*}

As is usual procedure when trying to derive a variational formulation, we multiply (\ref{linEl}) with some test function $\boldsymbol{v}\in V$, where the function space $V$ will be determined from what seems useful during the derivation. Then we integrate over the domain to get
\begin{equation}
\label{step1}
\int_{\Omega} (\nabla \sigma(\boldsymbol{u}))\cdot \boldsymbol{v}d\Omega = -\int_{\Omega}\boldsymbol{f}\cdot \boldsymbol{v} d\Omega .
\end{equation}
We aim at applying Green's formula to the left hand side, and since Green's formula states that
\begin{equation*}
\int_{\Omega}\nabla \cdot \boldsymbol{a}d\Omega = \int_{\partial \Omega} \boldsymbol{a}\cdot \boldsymbol{\hat{n}}dS 
\end{equation*}
we should come up with a good candidate for $\boldsymbol{a}$. A good guess seems to be $\boldsymbol{a} = \sigma(\boldsymbol{u})\boldsymbol{v}$. Now, this isn't exactly the form the integrand on the left in (\ref{step1}) takes, so we need to do some calculations and find a relation between the two. Disclaimer for the following calculation: It is long, tedious and not extremely easy to read. In addition, for convenience's sake we use the notation $\frac{\partial}{\partial x_i} = \partial_{x_i}$.
\begin{align*}
\nabla \cdot (\sigma(\boldsymbol{u})\boldsymbol{v}) &= \nabla \cdot \begin{bmatrix}
\sigma_{11}v_1 + \sigma_{12}v_2 \\
\sigma_{12}v_1 + \sigma_{22}v_2
\end{bmatrix}\\
&= \partial_{x_1}(\sigma_{11}v_1 + \sigma_{12}v_2) + \partial_{x_2}(\sigma_{12}v_1 + \sigma_{22}v_2) \\
&= v_1\partial_{x_1}\sigma_{11} + \sigma_{11}\partial_{x_1}v_1 + v_2\partial_{x_1}\sigma_{12} + \sigma_{12}\partial_{x1}v_2  \\
&+ v_1\partial_{x_2}\sigma_{12} + \sigma_{12}\partial_{x2}v_1 + v_2\partial_{x_2}\sigma_{22} + \sigma_{22}\partial_{x_2}v_2 .
\end{align*}
Up to this point we've just done a vector dot product and some differentiation using the product rule. Let's continue by rearranging the terms to
\begin{align*}
\nabla \cdot (\sigma(\boldsymbol{u})\boldsymbol{v})&= v_1(\partial_{x_1}\sigma_{11} + \partial_{x_2}\sigma_{12}) + v_2(\partial_{x_1}\sigma_{12} + \partial_{x_2}\sigma_{22}) \\
&+ \sigma_{11}\partial_{x_1}v_1 + \sigma_{12}(\partial_{x_2}v_1+\partial_{x_1}v_2)+\sigma_{22}\partial_{x_2}v_2 .
\end{align*}
We're definitely closing in on something. To make it even more easy on the eye we go back to the integrand on the left in (\ref{step1}) and see that
\begin{equation*}
(\nabla \sigma(\boldsymbol{u}))\cdot \boldsymbol{v} = v_1(\partial_{x_1}\sigma_{11} + \partial_{x_2}\sigma_{12}) + v_2(\partial_{x_1}\sigma_{12} + \partial_{x_2}\sigma_{22}),
\end{equation*}
which corresponds nicely with the first part of what we have after rearranging terms. The last part we deal with by looking back at how the strain vector was defined. Using these two things we end up with
\begin{equation}
\label{step2}
\nabla \cdot (\sigma(\boldsymbol{u})\boldsymbol{v}) = (\nabla \sigma(\boldsymbol{u}))\cdot \boldsymbol{v} + \boldsymbol{\varepsilon}(\boldsymbol{v})\cdot \boldsymbol{\sigma}(\boldsymbol{u}).
\end{equation}
The result of putting this into (\ref{step1}) is
\begin{equation*}
\int_{\Omega}\boldsymbol{\varepsilon}(\boldsymbol{v})\cdot \boldsymbol{\sigma}(\boldsymbol{u})d\Omega - \int_{\Omega}\nabla \cdot (\sigma(\boldsymbol{u})\boldsymbol{v})d\Omega = \int_{\Omega}\boldsymbol{f}\cdot \boldsymbol{v} d\Omega,
\end{equation*}
and we've put ourselves in position to employ Green's formula on the second term on the left hand side. What we end up with is
\begin{equation*}
\int_{\Omega}\boldsymbol{\varepsilon}(\boldsymbol{v})\cdot \boldsymbol{\sigma}(\boldsymbol{u})d\Omega = \int_{\Omega}\boldsymbol{f}\cdot \boldsymbol{v} d\Omega + \int_{\partial \Omega}(\sigma(\boldsymbol{u})\boldsymbol{v})\cdot\boldsymbol{\hat{n}}dS,
\end{equation*}
and can be further simplified by letting $V = H^1_{\Gamma_D}=\{v \in H^1 : v=0$ on $\partial \Omega_D \}$. Then
\begin{align*}
\int_{\Omega}\boldsymbol{\varepsilon}(\boldsymbol{v})\cdot \boldsymbol{\sigma}(\boldsymbol{u})d\Omega &= \int_{\Omega}\boldsymbol{f}\cdot \boldsymbol{v} d\Omega + \int_{\partial \Omega_N}\boldsymbol{v}\cdot \sigma\boldsymbol{\hat{n}}dS  \notag \\
&= \int_{\Omega}\boldsymbol{f}\cdot \boldsymbol{v} d\Omega + \int_{\partial \Omega_N}\boldsymbol{v}\cdot\boldsymbol{h}dS
\end{align*}

We are now prepared to make a variational formulation of (\ref{linEl}): 

Find $u \in \{H^1 : u=g,$ on $\partial\Omega_D\}$ so that
\begin{equation}
\label{VarForm}
\int_{\Omega}\boldsymbol{\varepsilon}(\boldsymbol{v})\cdot C\boldsymbol{\varepsilon}(\boldsymbol{u})d\Omega = \int_{\Omega}\boldsymbol{f}\cdot \boldsymbol{v} d\Omega + \int_{\partial \Omega_N}\boldsymbol{v}\cdot\boldsymbol{h}dS
\end{equation}
for all $v\in V$. Here we have now substituted $C\boldsymbol{\varepsilon}$ for $\boldsymbol{\sigma}$.

\subsection*{Galerking projection}
Since both function spaces we're looking at er infinite dimensional it is a good idea to look let the test function reside in the function space $X_h \subset V$ of piecewise linear functions on a triangulation of $\Omega$. Furthermore, let $X_h =$span$(\varphi_1,\ldots,\varphi_n)$ where $\phi_i$ are basis function with some compact support. Since the test functions $v$ are now vector functions, there will for each node $\hat{i}$ be two test functions
\begin{align*}
\varphi_{\hat{i},1} &= \begin{bmatrix}
\varphi_{\hat{i}} \\ 0
\end{bmatrix} \\
\varphi_{\hat{i},2} &= \begin{bmatrix}
0 \\ \varphi_{\hat{i}}
\end{bmatrix},
\end{align*}
and there should be some connection between the basis number $i$, node number $\hat{i}$ and $d=1,2$. If the total number of nodes are $N$, then total number of basis functions should be $2N$. A good function between the two can be $i = 2\hat{i}+ (d-2)$. Then $i$ ranges from $1$ to $2N$. To build up a linear system of what we have so far we state the Galerkin formulation of the problem: Find $u \in X_h$ so that
\begin{equation}
\label{Galerkin}
\int_{\Omega}\boldsymbol{\varepsilon}(\boldsymbol{v})\cdot C\boldsymbol{\varepsilon}(\boldsymbol{u})d\Omega = \int_{\Omega}\boldsymbol{f}\cdot \boldsymbol{v} d\Omega + \int_{\partial \Omega_N}\boldsymbol{v}\cdot\boldsymbol{h}dS
\end{equation}
for all $v\in X_h$.

It is sufficient to find a $u$ where this is true for all the basis frunctions since the right hand side is a bilinear form and the right hand side is a linear functional. That means we can let $\boldsymbol{v} = \varphi_i$ with $i\in\{1,2,\ldots,2N\}$ and $u = \sum_j u_j\varphi_j$. We put this into our Galerking formulation as
\begin{align*}
\int_{\Omega}\boldsymbol{\varepsilon}(\boldsymbol{\varphi_i})\cdot C\boldsymbol{\varepsilon}\left(\sum_j u_j\boldsymbol{\varphi_j}\right)d\Omega &= \int_{\Omega}\sum_j u_j\boldsymbol{\varepsilon}(\boldsymbol{\varphi_i})\cdot C\boldsymbol{\varepsilon}(\boldsymbol{\varphi_j})d\Omega \\
&=\sum_j u_j\int_{\Omega}\boldsymbol{\varepsilon}(\boldsymbol{\varphi_i})\cdot C\boldsymbol{\varepsilon}(\boldsymbol{\varphi_j})d\Omega \\
&= \int_{\Omega}\boldsymbol{f}\cdot \boldsymbol{\varphi_i} d\Omega + \int_{\partial \Omega_N}\boldsymbol{\varphi_i}\cdot\boldsymbol{h}dS.
\end{align*}
Taking this to be true over all $i$ we get the linear system
\begin{equation*}
A\boldsymbol{u} = \boldsymbol{b},
\end{equation*}
where
\begin{align*}
A_{ij} &= \int_{\Omega}\boldsymbol{\varepsilon}(\boldsymbol{\varphi_i})\cdot C\boldsymbol{\varepsilon}(\boldsymbol{\varphi_j})d\Omega \\
b_i &=  \int_{\Omega}\boldsymbol{f}\cdot \boldsymbol{\varphi_i} d\Omega + \int_{\partial \Omega_N}\boldsymbol{\varphi_i}\cdot\boldsymbol{h}dS.
\end{align*}
This linear system is what we will be implementing.

\subsection*{A simple test case}
When using a finite element implementation it's in most cases a sound strategy to have compared to a test case where the exact solution is known. To build some confidence in the implementation. As a test example we are going to consider the following:
\begin{equation}
\label{testCase}
\begin{cases}\nabla \boldsymbol{\sigma}(\boldsymbol{u}) &= -\boldsymbol{f}, \quad \text{in } \Omega \\
\boldsymbol{u} &= \boldsymbol{0} \quad \text{on } \partial \Omega
\end{cases}
\end{equation}
where $\Omega = \{(x,y)\in \mathbb{R}^2: |x|,|y|<1\}$ and $\boldsymbol{f}$ is given as
\begin{align*}
f_1 &= \frac{E}{1-\nu^2}\left(-2y^2-x^2+\nu x^2-2\nu xy - 2xy + 3 - \nu\right) \\
f_2 &= \frac{E}{1-\nu^2}\left(-2x^2-y^2+\nu y^2-2\nu xy - 2xy + 3 - \nu\right).
\end{align*}
In addition we are given
\begin{equation*}
\boldsymbol{u} = \begin{bmatrix}
(x^2-1)(y^2-1) \\ (x^2-1)(y^2-1)
\end{bmatrix},
\end{equation*}
and all we need to do is verify that this is in fact the solution of (\ref{testCase}). So let's do just that. To begin with we notice easily that the boundary condition is satisfied, so we proceed by calculating the strains as
\begin{equation*}
\begin{bmatrix}
\varepsilon_{11} \\
\varepsilon_{22} \\
\varepsilon_{12}
\end{bmatrix}
=
\begin{bmatrix}
2x(y^2-1) \\
2y(x^2-1) \\
2y(x^2-1)+2x(y^2-1)
\end{bmatrix}.
\end{equation*}
Now we make use of the affine relation between $\sigma$ and $\varepsilon$ to deduce that
\begin{align*}
\begin{bmatrix}
\sigma_{11} \\
\sigma_{22} \\
\sigma_{12} \\
\end{bmatrix}
&= \frac{E}{1-\nu^2}\begin{bmatrix}
\varepsilon_{11} + \nu \varepsilon_{22} \\
\nu \varepsilon_{11} + \varepsilon_{22} \\
\frac{1-\nu}{2}\varepsilon_{12}
\end{bmatrix}
\\
&=\frac{E}{1-\nu^2}\begin{bmatrix}
2x(y^2-1) + 2\nu y(x^2-1) \\
2\nu x(y^2-1) +2y(x^2-1) \\
(1-\nu)(y(x^2-1)+x(y^2-1))
\end{bmatrix}
\end{align*} 
What follows is a tedious calculation, so brace yourself. The gradient of the stress is
\begin{align*}
\nabla \cdot \sigma(\boldsymbol{u}) &= \begin{bmatrix}
\partial_x \sigma_{11} + \partial_y \sigma_{12} \\
\partial_x \sigma_{12} + \partial_y \sigma_{22} 
\end{bmatrix} \\
&= \frac{E}{1-\nu^2}\begin{bmatrix}
2(y^2-1)+4\nu xy + (1-\nu)(x^2-1+2xy) \\
(1-\nu)(2xy + y^2-1)+4\nu xy +2(x^2 -1)
\end{bmatrix} \\
&= \frac{-E}{1-\nu^2}\begin{bmatrix}
-2y^2 - x^2 + \nu  x^2 - 2\nu xy - 2xy + 3 - \nu \\
-2x^2-y^2 + \nu y^2 - 2\nu xy -2xy + 3 -\nu
\end{bmatrix} \\
&= -\boldsymbol{f}
\end{align*}

\end{document}